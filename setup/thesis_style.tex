\usepackage[utf8]{inputenc}

% Handle multi-glyph characters in the output PDF correctly
\usepackage[T1]{fontenc}

% Main font with math support
\usepackage{charter}
% Sans serif font
\usepackage[scaled=.88]{berasans}
% Monospace font (for code)
\usepackage[scaled=.82]{DejaVuSansMono}

\usepackage{microtype}

\usepackage[english]{babel}
\usepackage{datetime}

\usepackage[backend=biber,style=bwl-FU,bibstyle=authoryear]{biblatex}

\usepackage{tikz}                               % Draw graphics
\usepackage{parskip}
\usepackage[colorinlistoftodos,prependcaption,textsize=tiny,textwidth=30mm]{todonotes}
\usepackage{xcolor}                             % Colored text etc.
\usepackage{caption}
\usepackage{subcaption}
\usepackage{graphicx}							% To include graphics
\usepackage{csquotes}
\usepackage{fancyhdr}						    % Nice header and footer
\usepackage{amsmath}                            % Mathematical typesetting
\usepackage{amsfonts}
\usepackage{siunitx}
\usepackage{geometry}							% Better geometry
\usepackage{titlesec}							% change section headings
\usepackage{booktabs}							% Nice tables
%\usepackage[center]							% For cropping documents
\usepackage[linktocpage,colorlinks]{hyperref}	% PDF hyperlink
\usepackage{float}                              % repositioning floats
\usepackage{array}
% todo notes related
\usepackage{xargs}                              % Use more than one optional parameter in a new commands
\usepackage[bottom]{footmisc}

\usepackage{framed} % leftbar
\usepackage{textpos}
% Custom list formatting, such as noitemsep and label
\usepackage{enumitem}
% Itemize and enumerate: tighter and with other symbols
\setlist[enumerate]{itemsep=0mm, topsep=5pt, partopsep=0mm, parsep=0mm}
\setlist[enumerate,1]{label=\arabic*., ref=\arabic*}
\setlist[enumerate,2]{label=\alph*., ref=\alph*}
\setlist[enumerate,3]{label=\roman*., ref=\roman*}
\setlist[itemize]{itemsep=0mm, topsep=5pt, partopsep=0mm, parsep=0mm}
\setlist[itemize,1]{label=$\bullet$}
\setlist[itemize,2]{label=$\circ$}
\setlist[itemize,3]{label=$-$}

\usepackage{wrapfig}


\geometry{
a4paper,
margin=1.5in
}

% B5 (uncomment to convert to B5 format)


% Colors
\definecolor{nicegray}{gray}{.25}
\definecolor{nicegreen}{HTML}{0F7F12}
\definecolor{poaceae}{HTML}{1f77b4}
\definecolor{corylus}{HTML}{ff7f0e}
\definecolor{alnus}{HTML}{2ca02c}

% Captions
\captionsetup{labelfont=bf,textfont={small,color=nicegray},margin=\parindent}

% date format
\newdateformat{ddaymonth}{%
\ordinaldate{\THEDAY} \monthname[\THEMONTH]}
\newdateformat{daymonth}{%
\twodigit{\THEDAY}\ \monthname[\THEMONTH]}

%%%
% Author
% Fill in here, and use commands in the text. 
\newcommand{\thesisAuthor}{Fredrik Gyllenhammar}
\newcommand{\thesisTitle}{Automated Pollen Grain Counting}
\newcommand{\thesisType}{Thesis project}
\newcommand{\thesisDate}{fall 2020}

% PDF info
\hypersetup{
    pdfauthor={\thesisAuthor},
    pdftitle={\thesisTitle},
    pdfsubject={\thesisType},
    linkcolor={black},
    citecolor={black},
    urlcolor={black},
}

% siunitx
\DeclareSIUnit\pixel{pixel}

% TODO notes
\newcommandx{\info}[2][1=]{\todo[linecolor=Green,backgroundcolor=Green!25,bordercolor=Green,#1]{#2}}

% Heading format 
%\titleformat{\chapter}{\bfseries\LARGE}{\thechapter}{1em}{}
\titleformat*{\section}{\Large\bfseries}
\titleformat{\subsection}{\large\bfseries}{}{0em}{}

\titlespacing*{\chapter}{0cm}{2in}{1ex}

% Bibliography

\addbibresource{bibtex/zotero.bib}
\setcounter{biburllcpenalty}{7000}
\setcounter{biburlucpenalty}{8000}

% Environments
% Quick and dirty pseudocode which does not deserve the algorithm environment
\newenvironment{pseudocode}[0]
{%
  \begin{leftbar}
  \noindent
  \hspace{-0.6em}
}
{%
  \end{leftbar}
  \noindent
  \hspace{-0.8em}
}
\newenvironment{pseudoloop}[0]
{%
  \begin{itemize}[nosep,label=,leftmargin=1em]
}
{%
  \end{itemize}
}
\newenvironment{pseudofunc}[2]
{%
  \begin{pseudocode}
  procedure \textbf{\texttt{#1}}(#2):
  \begin{pseudoloop}
}
{%
  \end{pseudoloop}
  \end{pseudocode}
}
\newcommand{\pmethod}[2]{\texttt{#1(#2)}}
\newcommand{\parg}[2]{\(\texttt{#1}=#2\)}
\newcommand{\pif}{\textbf{if}}
\newcommand{\pwhile}{\textbf{while}}
\newcommand{\pfor}{\textbf{for}}
\newcommand{\pret}{\textbf{return}}