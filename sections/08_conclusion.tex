\chapter{Conclusion}\label{cha:conclusion}
In regards to the two initial research questions posed in this thesis and based upon the results of the experiments presented in Chapter~\ref{cha:results}, the following is given as the main conclusions of this work,

\begin{enumerate}
    \item For the task of differentiating pollen grains, reducing the size of the network negatively affects precision. However, the uniformity of the objects in the dataset allows for a reduction in the number of detections made by the model without any apparent performance effects.
    \item Excluding multifocal data from training causes a fixation on sharp features, inhibiting the model from localizing pollen grains only slightly outside the focal range of the training data.
    \item Excluding multifocal data creates a more precise model with lower recall than a model trained on multifocal data.
\end{enumerate}

The solution presented is a fully convolutional deep neural network capable of locating and classifying pollen grains from microscopic imaging data from a standard stereoscopic microscope.
The performance of the presented model supports the more general claim that locating pollen is well suited for a CNN based solution.
This first attempt using a CNN model shows that both the task of localizing and classifying pollen grains is solvable with a fully convolutional model.
An adapted SSD model was used for this project, but future work may find that other single shot models, such as YOLO could provide better results.
The trained model is by itself not a major contribution of this work but a proof of concept of the general approach.

The limits of this method are unknown, but the results presented indicate that localization is a far simpler task than classification, which could be the limiting factor in the scaling of this model to a wider set of classes.

This work makes two main contributions to the joint fields of palynology and machine learning:

\begin{enumerate}
    \item A new, relatively large, pollen detection dataset totaling 701 sample images and 6384 ground truth labels of Norwegian air-born pollen. As of writing, a dataset with both localizations and class labels has not been presented in the literature.
    \item Evidence showing a benefit to using un-sharp data in the training of convolutional object detection models in domains using microscopic imaging data.
\end{enumerate}

\section{Future Work}
The most important contribution is in reference to the use of unfocused data when training detection models, which to the knowledge of the author, is a novel discovery.
Based on this and the weaknesses of this thesis, multiple paths can be explored in future work, which builds on this work.

\subsection*{Extending the dataset}
The size of the dataset could pose a threat to the external validity of the model's performance.
The results show that the model performs better at localization than at classification in a dataset containing three distinct classes.
Within a domain where most potential classes share similar features, it is unknown how the model would perform with a dataset containing more classes and if it could maintain its ability to label classes correctly.
Therefore, it cannot be concluded that the model would perform at a similar level of precision in datasets with more classes.
Extending the number of classes in the dataset with more samples is the only way to verify this.

A different way of extending the existing dataset is by using synthesized data, which could significantly improve performance.
New sample images could be artificially generated by rearranging ground truths and moving them in the focal plane.

\subsection*{Multifocal input}
The results assume a link between perceived sharpness and focal planes in the sample images without a direct correlation being shown.
This perceived sharpness is used to validate the sharpness measure, which in turn is used to analyze the model in relation to data sharpness.
To an extent, this assumption does hold; when the focal plane lies above or below a pollen grain, perceived sharpness changes with the focal plane.
However, on a more granular level, how the sharpness measure corresponds to the location of the focal plane when it lies within a pollen grain is less clear.
From observation, maximum sharpness does seem to occur at the center of the pollen grain when the diameter of the grain is largest.

Using stacks of images from multiple focal planes as the input to the model, instead of a single image, could enhance performance by providing the model a detailed view of the entire surface of each pollen grain.

\subsection*{Counting algorithm}
Turning the model's raw output into an accurate pollen grain count for an entire microscopic slide is a non-trivial task.
Pollen grains may be placed directly over one another, only distinguishable by running the model over all focal planes.
One possible approach is to use video streams from a microscope, taken while it slides its focal plane across a microscope slide and, from all the individual detections and sharpness values, create a 3D positional model of every detected grain.
This would enable counting in three dimensions and could require the creation of a novel filtering algorithm that extends the concept of overlapping predictions into three dimensions.

\subsection*{Live counting}
A second possibility for a functioning counting system is to embed the model in a live system that controls the motion of a microscope along all three axes, similar to the current manual method with human operators.
The model's output could be used to guide a new search algorithm that moves over a slide, building a complete detection model for the entire slide in all three axes.

