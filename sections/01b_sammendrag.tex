%% spellcheck-off

\chapter*{Sammendrag}
Denne oppgaven utforsker hvodan CNN baserte objektdeteksjonsmodeler kan bruker til å lokalisere of klassifisere pollenkorn ved hjelp av mikroskopisk bilde data.
Telling av pollen er en sentral metode innen mange forskellige felt, f.eks.\ krimimalogi, arkeologi, og geologi.
Dette er en møysommelig og veldig tidkrevende oppgave som per nå krever ekspertkunnskap.
Fra litteraturen finnes det åpne spørsmål med hensyn til kompleksiteten som trengs for å løse dette problemet i forhold til mer vanlige objektdeteksjonsoppgaver.
Effekten skarpheten til treningsekemplene her på modellen er også uklar.
Eksperimenter med en `Single Shot Multibox' deteksjonsmodell viser at problemet er løselig med en fullt konvolusjonell modell.
Den regulære formen til pollenkorn tillater visse forenklinger av modellen, men likhetene på tvers av klassene fører til tap av nøyaktighet i mindre modellkonfigurasjoner.
Ekskludering av uskarpe data fra modellopplæringen får modellen til å fiksere på skarphet, noe som reduserer modellens evne til å identifisere korn som er mindre skarpe en trenings eksemplene.
Trening med uskarpe eksempeler ser ut til å tillate en mer robust generalisering over de ukile attributtene i multifokale data.