\chapter*{Abstract}
This thesis explores how a CNN based object detection model may be used to localize and classify pollen grains using microscopic imaging data.
Pollen counting is a central method in many diverse fields, e.g.\ criminology, archaeology, and geology.
This is a laborious and very time-consuming task which currently requires expert knowledge.
From literature, open questions remain with regards to the complexity needed to solve this problem versus more common object detection tasks
The effects of sharpness within training examples is also unclear.
Experiments using a Single Shot Multibox Detection model reveal that the problem is solvable with a fully convolutional model. 
The regular shape of pollen grains allows for certain simplifications to the model, but the similarities across classes cause a loss in accuracy in smaller model configurations.
Excluding un-sharp data from the model training causes the model to fixate on sharpness which reduces the models ability to identify grains that appear less sharp.
Training with un-sharp example seems to allow for a more robust generalization over the features encoded in multifocal data.

%% spellcheck-off

\chapter*{Sammendrag}
Denne oppgaven utforsker hvodan CNN baserte objektdeteksjonsmodeler kan bruker til å lokalisere of klassifisere pollenkorn ved hjelp av mikroskopisk bilde data.
Telling av pollen er en sentral metode innen mange forskellige felt, f.eks.\ krimimalogi, arkeologi, og geologi.
Dette er en møysommelig og veldig tidkrevende oppgave som per nå krever ekspertkunnskap.
Fra litteraturen finnes det åpne spørsmål med hensyn til kompleksiteten som trengs for å løse dette problemet i forhold til mer vanlige objektdeteksjonsoppgaver.
Effekten skarpheten til treningsekemplene her på modellen er også uklar.
Eksperimenter med en `Single Shot Multibox' deteksjonsmodell viser at problemet er løselig med en fullt konvolusjonell modell.
Den regulære formen til pollenkorn tillater visse forenklinger av modellen, men likhetene på tvers av klassene fører til tap av nøyaktighet i mindre modellkonfigurasjoner.
Ekskludering av uskarpe data fra modellopplæringen får modellen til å fiksere på skarphet, noe som reduserer modellens evne til å identifisere korn som er mindre skarpe en trenings eksemplene.
Trening med uskarpe eksempeler ser ut til å tillate en mer robust generalisering over de ukile attributtene i multifokale data.
